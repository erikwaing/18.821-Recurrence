\title{Fermi's Thoughts}
\date{}

\documentclass[12pt]{article}
\usepackage{amsmath}
\usepackage{amsthm}

\begin{document}
\maketitle


\section{Period 2 for $F(x,y)$}

Suppose we have period 2. Then our sequence is $x,y,x,y,\dots$, and in particular, we know that $F(x,y) = x$ must hold by looking at the third term.

Since this equality holds, that is the only possible value for $F(x,y)$. Also, since $F(x,y)$ is not of the form $f(y)/x$, this shows that there are no period 2 sequences generated by $f(y)/x$.

\section{Period 3 for $f(y)/x$}

Suppose we have period 3. Then our sequence is
\[ x,y,f(y)/x,x,y,f(y)/x,\dots\]
and from looking at the fourth and fifth terms, we get the following two equations
\begin{align*}
x &= f(f(y)/x)/y\\
y &= xf(x)/f(y)\\
\end{align*}
Note that the second equation tells us that $yf(y) = xf(x)$, and this must hold true for any choice of $x$ and $y$, and so this mean that $yf(y) = xf(x) = c$ for some constant $c$. This tells us that $f(y)$ must be of the form $c/y$.

So $\frac{c}{xy}$ is the only function of the form $f(y)/x$ with period 3.

\section{Period 4 for $f(y)/x$}

Assume $f(y)$ is a nonconstant polynomial. Suppose we have period 4. We show that we will get a contradiction.

Our sequence is
\[ x,y,\frac{f(y)}{x},\frac{f(f(y)/x)}{y},x,y,\frac{f(y)}{x},\frac{f(f(y)/x)}{y},\dots\]

Look at the sixth term, and note that the following relation must hold
\[ y = \frac{f(x)y}{f(f(y)/x)} \]
or equivalently if $y /neq 0$,
\[ f(f(y)/x) = f(x) \]
Now fix $x = 1$. Then we get 
\[ f(f(y)) = f(1) = c \]

Note that since this equation must hold, and since $f(y)$ can take on infinitely many values (this is true of any nonconstant polynomial), we have infinitely many solutions to $f(f(y)) = f(1)$. In particular, $f(f(y)) - f(1)$ is a polynomial in $f(y)$ with infinitely many distinct roots. This is a contradiction.

So for this case to work, $f(y)$ must be a constant.

\section{Period 5 for $f(y)/x$}

We claim that anything of the form $f(y) = ay + a^2$ gives period 5. Using this function, we have the following sequence
\[ x, y, \frac{a^2+ay}{x}, \frac{a^3 + a^2x + a^2y}{xy}, \frac{a^2 + ax}{y}, x, y, \dots \]


\end{document}
