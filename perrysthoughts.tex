\title{Perry's Thoughts}
\date{}

\documentclass[12pt]{article}
\usepackage{amsmath}
\usepackage{amsthm}

\begin{document}
\maketitle
\section{How to Deal with 0}
In order for our rules to be complete we must have some method for dealing with $0$. In order to handle this we will include an inverse for $0$ which we will write as $0^{-1}$. So we have 
\begin{align*}
0*0^{-1} =1
\end{align*}
We will treat $0^{-1}$ as an variable, that is related to the reals only by the above relation. We are essentially ignoring $0$ and/or when working with our equations we are assuming that the denominator is never $0$.

\section{Question 4} 
I hypothesize that $f(y) = by+c, b \neq 0$ is periodic for $c = 1, b=1$ with period 5, and is not periodic when $c \neq 0,1$ or when $b \neq 1$. That is the only linear polynomials which are periodic are $y+1$ and $by$ for $ b \neq 0$. 
\begin{proof}
If $c=1$ suppose we have $a_0 =x, a_1 =y$, we want to show that $a_5=a_0$ and $a_6=a_1$. Well 
\begin{equation*}
a_2 = \frac{f(a_1)}{a_0} = \frac{ a_1 +1}{a_0} = \frac{y+1}{x},
\end{equation*}
and so 
\begin{equation*}
a_3 = \frac{a_2+1}{a_1} = \frac{ \frac{y+1}{x} +1}{y} = \frac{x+y+1}{xy},
\end{equation*}
so
\begin{equation*}
a_4 = \frac{a_3 +1}{a_2} = \frac{ \frac{y+1+x}{xy} +1}{\frac{y+1}{x}} = \frac{x+y+1+xy}{y(y+1)},
\end{equation*}
giving us 
\begin{align*}
a_5 &=\frac{a_4+1}{a_3} = \left(\frac{x+y+1+xy}{y(y+1)} +1\right)\left( \frac{xy}{x+y+1} \right)= \frac{ x(x+y+1) + xxy + xy(y+1)}{(y+1)(x+y+1)} \\
&= \frac{x^2(y+1) + x(y+1) + xy(y+1)}{(y+1)(x+y+1)} = \frac{x(x+y+1)}{x+y+1} = x = a_0.
\end{align*}
So finally we calculate $a_6$ and find 
\begin{align*}
a_6 = \frac{a_5 +1}{a_4} = (x+1)\left( \frac{y(y+1)}{x+y+1+xy} \right) = \frac{xy^2 + xy+y^2+y}{x+y+1+xy} =y =a_1
\end{align*}
Therefore $f(y)=y+1$ is periodic with period 5 as desired. 
\end{proof}
By numerical simulation we have checked that if $f(y) = y+2$ or $f(y)=y-1$ are periodic they have period longer than 40,000. 

\section{Linear Rational Functions}
In this section we consider $F(x,y)$ of the form
\begin{equation}
\label{rationallinear}
F(x,y) = \frac{b_1 x+b_2 y + b_3}{c_1 x + c_2 y+ c_3},
\end{equation}
and ask the questions given $n$ for what choice of $b_i$ and $c_i$ will we produce a function with a period $n$ and for what $n$'s can we do this?

It is somewhat obvious that $F(x,y)=x$ is the only ratio of linear functions that has period 2, however for larger $n$ the question and answer are slightly more interesting. 
\begin{proposition} 
If $F(x,y)$ is of the form in \ref{rationallinear} and the recurrence relation it produces has period $3$ then $F$ is one of three functions. In particular we can have 
\begin{align*}
F(x,y) &= -x-y \\
F(x,y) &= \left( \frac{1}{2} +\frac{i\sqrt{3}}{2} \right) x + \left( \frac{1}{2} -\frac{i\sqrt{3}}{2} \right) y \\
F(x,y) &= \left( \frac{1}{2} -\frac{i\sqrt{3}}{2} \right) x + \left( \frac{1}{2} +\frac{i\sqrt{3}}{2} \right) y
\end{align*}
\end{proposition}
\begin{proof}
In order for $F(x,y)$ to induce a recurrence relation with period $3$ we must have
\begin{equation*}
\left( (c_1 x + c_2 y +c_3) c_1 y + c_2 (b_1 x + b_2 y + b_3) + c_3(c_1 x + c_2 y +c_3) \right) x = (c_1 x + c_2 y + c_3) b_1 y + b_2(b_1 x + b_2 y + b_3) + b_3 (c_1 x = c_2 y + c_3),
\end{equation*}
which we obtain from writing down $a_3=a_0$ in terms of $x$ and $y$ and clearing denominators. If we multiply this expression out we get 
\begin{equation*}
c_1^2 x^2 y + c_2 c_1 y^2 x + (c_3 c_1 + c_2 b_2 + c_3 c_2)xy + (c_2b_1 + c_3 c_1) x^2 + (b_3 + c_3)x = c_1 b_1 xy + c_2 b_1 y^2 + c_3 b_1 y + (b_2 b_1 + b_3 c_1)x + (b_2^2 +b_3 c_2)y + b_2 b_3 + b_3 c_3.
\end{equation*}
Because this relation must hold for arbitrary $x$ and $y$ we know that terms with the same exponents for $x$ and $y$ must cancel. So for instance we must have 
\begin{equation*}
c_1^2 x^2 y = 0, 
\end{equation*}
and so we must have $c_1=0$. We can do this for terms of each order and find the following 5 relations for the coefficients 
\begin{align}
\label{fiverelations}
c_2(b_2 + c_3) =0 \\
c_2 b_1 = 0 \\
b_3 + c_3 = b_2 b_1 \\
c_3 b_1 + b_2^2 + b_3 c_2 =0 \\
b_3(b_2 + c_3) = 0. 
\end{align}
Suppose that $c_2 =0$, then $F$ is of the form 
\begin{equation*}
F(x,y) = \frac{ b_1 x + b_2 y + b_3}{c_3}.
\end{equation*}
Therefore we must have 
\begin{equation*}
c_3^2 x = b_1 c_3 y + b_2 (b_1 x + b_2 y + b_3) + b_3,
\end{equation*}
which gives us a new family of relations 
\begin{align*}
b_2 b_1 = c_3^2 \\
c_3 b_1 + b_2^2 =0.
\end{align*}
Combining these two gives us 
\begin{equation*}
c_3^3 + b_2^3 = 0, 
\end{equation*}
which has three solutions $c_3 = -b_2$ and $c_3 = \frac{1}{2} ( b_2 \pm i\sqrt{3}b_2)$. Note that $c_3 = -b_2$ implies that we must have $b_1=b_2$ which gives us $F(x,y) = -x -y$. It is easy to check that this is in fact periodic with period 3. On the other hand $c_3 = \frac{1}{2} ( b_2 \pm i\sqrt{3}b_2)$ implies that $b_1 = \frac{1}{2} ( -b_2 \pm i\sqrt{3}b_2)$, which gives us 
\begin{align*}
F(x,y) &= \left( \frac{1}{2} -\frac{i\sqrt{3}}{2} \right) x + \left( \frac{1}{2} +\frac{i\sqrt{3}}{2} \right) y \\
F(x,y) &= \left( \frac{1}{2} +\frac{i\sqrt{3}}{2} \right) x + \left( \frac{1}{2} -\frac{i\sqrt{3}}{2} \right) y.
\end{align*}
It is also easy to check that these two $F$ have period 3. This exhausts the possibilities when $c_2=0$ so let us asssume otherwise. Then based on the relations \ref{fiverelations} we must have 
\begin{equation*}
b_1 = 0,
\end{equation*}
and so $F(x,y)$ takes the form 
\begin{equation*}
F(x,y) = \frac{b_2 y + b_3}{c_2 y + c_3}.
\end{equation*}
Therefore if $a_3 = a_0$ we must have that 
\begin{equation*}
x = \frac{ b_2^2 y + b_2b_3 +b_3(c_2 y + c_3)}{c_2 b_2 y + c_2 b_3 + c_3 c_2 y + c_3^2},
\end{equation*}
but the left hand side depends only on $x$ and the right hand side depends only on $y$. This relation cannot hold for arbitrary $x,y$ and so we cannot have such $F$. Therefore we must have $c_2=0$ and so the only $F$ with period 3 are the ones we have given. 
\end{proof}
\end{document}
