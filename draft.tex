
\title{A Recurrence}
\author{
  Perry Kleinhenz \and
  Fermi Ma \and
  Erik Waingarten
}
\date{\today}

\documentclass[12pt]{article}
\usepackage{amsmath}
\usepackage{amsthm}

\newtheorem{proposition}{Proposition}

\begin{document}
\maketitle

\section{Introduction}
\emph{Written by Erik Waingarten}

In this paper, we will study recurrences with \textit{memory two}. By memory two, we mean that the sequence can be described by a recurrence where a certain number depends on the previous two numbers. We can define a sequence $a_0, a_1, a_2, \dots$ recursively using a function $F(x,y)$ of two variables, where we assign
\[ a_{n+1} = F(a_n, a_{n-1}) \]
Note that this is the ultimate generalization of recurrences which depends on the previous two terms. The famous Fibonacci sequence fits this mold: we can let $F(x,y) = x + y$, and the recurrence will yield
\[ a_{n+1} = a_n + a_{n-1} \]
If we let $a_0 = a_1 = 1$, we get the Fibonacci sequence. In this paper, we will focus on the properties that arise out of different definitions of the functions $F(x,y)$, independent of the intial terms. So usually, we will let $a_0 = x$ and $a_1 = y$, and we assume that $x$ and $y$ can take on any values. We will say that a recursive definition is \textit{periodic} if, with indeterminate intiail values $a_0 = x$ and $a_1 = y$, there is an integer $n>0$ such that $a_n = x$ and $a_{n+1} = y$. The \emph{period} is the smallest such $n$. 

We can compute some simple examples to promote our search. 
\begin{itemize}
\item $F(x,y) = y$: $x, y, x, y, x, \dots$, so we get period $2$.
\item $F(x,y) = \dfrac{1}{xy}$: $x, y, \dfrac{1}{xy}, x, y, \dots$, so we get period $3$.
\item $F(x,y) = \dfrac{1}{x}$: $x, y, \dfrac{1}{x}, \dfrac{1}{y}, x, y, \dots$, so we get period $4$.
\end{itemize}

There are some functions $F(x,y)$ such as the one generating the Fibonacci sequence which do not yield periodic sequences. One can ask, is there a simple test that determines whether a function yields a periodic sequence? Can we characterize the such functions? What are the possible periods?

These are questions that we hope to answer. 

\section{The case of $F(x,y) = f(y)/x$}

We begin by studying the special case where $F(x,y) = \dfrac{f(y)}{x}$. So that means that the sequences are defined by the relation
\begin{equation}
\label{eq:firstcase}
a_{n+1} = \dfrac{f(a_{n-1})}{a_n}
\end{equation}
In the introduction, we've already seen two functions in this form that are periodic. Namely, $f(y) = 1$ and $f(y) = \frac{1}{y}$. 

\begin{proposition}
The only possible $f$ which yields a sequence of period $3$ according to the recurrence in Equation~\ref{eq:firstcase} is $f(y) = \dfrac{c}{xy}$.
\end{proposition}

\begin{proof}
We can look at the first few values of the sequence:
\[ x, y, \dfrac{f(y)}{x}, x, y, \dots \]
By looking at $a_4$, observe that
\[ y = \dfrac{xf(x)}{f(y)} \]
and so,
\[ y f(y) = x f(x) \]
Fix $x = 1$. Then we have $y f(y) = f(1)$, and so $f(y) = \dfrac{c}{y}$, where $c$ is any constant .
\end{proof}

Likewise, we can do a similar analysis for period $4$. 
\begin{proposition}
The only possible $f$ (polynomial) which yields a sequence of period 4 according to the recurrence in Equation~\ref{eq:firstcase} is $f(y) = c$. 
\end{proposition}

\begin{proof}
First, we can easily check that these functions yield a sequence with period $4$. If we had any function $f$ which yielded a sequence with period $4$, then 
\[ x, y, \dfrac{f(y)}{x}, \dfrac{f(f(y)/x)}{y}, x, y, \dots \]
And by looking at $a_6$, observe
\[ y = \dfrac{yf(x)}{f(f(y)/x)} \]
and so 
\[ f(f(y)/x) = f(x) \]
If we fix $x = 1$, then we have
\[ f(f(y)) = f(1) \]
Which means that $f(f(y))$ is a constant for every $y$. And so since $f(y)$ is non-constant, it will take on infinitely many values, which means that $f(y) - f(1)$ is also a polynomial with infinitely many roots, and so $f(y) = f(1)$. 
\end{proof}


\end{document}